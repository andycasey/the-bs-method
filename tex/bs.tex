
%@arxiver{diagram.pdf,figure3.pdf}

\documentclass[onecolumn]{aastex62}
\usepackage{amssymb}
\usepackage{graphicx}
\usepackage{color}
\usepackage{enumerate}

\newcommand{\vdag}{(v)^\dagger}
\newcommand\aastex{AAS\TeX}
\newcommand\latex{La\TeX}

\newcommand{\SampleSize}{2,330} 

\newcommand\lamost{LAMOST}
\newcommand\apogee{APOGEE}
\newcommand\rave{RAVE}
\newcommand{\project}[1]{\emph{#1}}
\newcommand{\gaia}{\project{Gaia}}
\newcommand{\Gaia}{\project{Gaia}}
\newcommand{\rp}{\textsl{rp}}
\newcommand{\bp}{\textsl{bp}}

\newcommand{\kepler}{\project{Kepler}}
\newcommand{\ktwo}{\project{K2}}
\newcommand{\logg}{\log_{10}[g\,({\rm cm\,s}^{-2})]}
\newcommand{\likelihood}{\mathcal{L}}

%\graphicspath{{figures/}}


\received{January X, 2019}
\revised{January X, 2019}
\accepted{January X, 2025}
\submitjournal{AAS Journals}

\shorttitle{The B-S method}
\shortauthors{B-S}


\begin{document}

\title{The B-S method}

\correspondingauthor{}
\email{}


\begin{abstract}
\end{abstract}

\keywords{BS}

\section{Introduction} \label{sec:intro}

Spectrographs give an absolute count on the number of electrons at
a given wavelength, but astronomers are often interested in the 
counts relative to the level of continuum emission of the source.
As a result, considerable effort is placed on estimating the source
continuum from nearby pixels where the counts are or low-order
polynomial. Few people care about this.


\section{Model} \label{sec:model}

% Data
The data are an array of $N$ pixels, each of which has a
dispersion value $x_{n}$, the flux counts $y_{n}$ and the 
uncertainty on that flux counts $\sigma_{yn}$.
We assume the data are drawn from a two-component mixture model:
a foreground model to describe the properties we care about, and
a background model to account for nuisance properties.
In stellar spectroscopy we are often interested in the strength
of an electronic transition in absorption relative to the continuum,
so the foreground component will simultaneously model the local 
continuum and model the absorption line of interest as a gaussian 
profile. We assume that the foreground model $f(x)$ has the form
\begin{equation}
	f(x) = {C}(x)\,\left[1 - A\exp\left(-\frac{(x - \mu_f)^2}{2\sigma^2_{f}}\right)\right]
\end{equation}
\noindent{}where $A$ is the amplitude of the absorption
profile, $\mu_f$ is the line centroid, $\sigma_f$ is the
standard deviation of the absorption profile, and $C(x)$ is
the local continuum evaluated at $x$. Here we take the local
continuum to be a second order polynomial:
\begin{equation}
	C(x) = c_{0}x^2 + c_{1}x^1 + c_2 \quad .
\end{equation}


We assume that all pixels are uncorrelated such that the
probability distribution function $p\left(y_n|x_n,\sigma_{yn},\theta\right)$
for the flux in the $n$th pixel $y_{n}$ is
\begin{equation}
	p_{f}\left(y_n|x_n,\sigma_{yn},A,\mu,\sigma_f,c_0,c_1,c_2\right) = \frac{1}{\sqrt{2\pi\sigma_{yn}^2}} \exp\left(-\frac{\left[y_{n} - f(x_n)\right]^2}{2\sigma_{yn}^2}\right) \quad .
\end{equation}



The role of the background model is to describe the neighbouring absorption
that we are not interested with. One approach could be to fit additional
profiles to the neighbouring absorption, but this becomes unwieldily.
Instead we assume the background model to be a lognormal distribution
with a mean
\begin{equation}
	\mu_{gn} = \frac{f(x_n) - y_n}{f(x_n)}
\end{equation}
\noindent{}such that the probability distribution function is
\begin{equation}
	p_{g}\left(y_n|x_n,\sigma_{yn},\theta,\sigma_g\right) = \frac{1}{y_n\sqrt{2\pi\sigma^2_{g}}}\exp\left(-\frac{\left[\ln{y_n}-\mu_{gn}\right]^2}{2\sigma^2_g}\right) \quad .
\end{equation}

We introduce $Q$ as a relative mixing parameter for the foreground and background models, and
denote ${\theta = \{A,\mu_f,\sigma_f,c_0,c_1,c_2,\sigma_g,Q\}}$, then the 
likelihood $\likelihood\left(y|x,\sigma_y,\theta\right)$ is proportional to the weighted
sum of the foreground and background models:
\begin{eqnarray}
	\likelihood \propto \prod_{n=1}^{N} \left[\frac{1 - Q}{\sqrt{2\pi\sigma^2_{yn}}}\exp\left(-\frac{\left[y_{n} - f(x_n)\right]^2}{2\sigma^2_{yn}}\right) + \frac{Q}{y_{n}\sqrt{2\pi\sigma^2_{g}}}\exp\left(-\frac{\left[\ln{y_n}-\mu_{gn}\right]^2}{2\sigma^2_g}\right)\right] \quad .
\end{eqnarray}

\the\textwidth

\subsection{Toy model}

Here we provide some intuition on the parameters of the mixture model. We take a model
of the Solar spectrum that is continuum normalised and randomly generate the following 
parameters:
\begin{eqnarray}
	c_0 = +1.26\times10^1 \\
	c_1 = -1.35\times10^5 \\
	c_2 = +3.62\times10^8
\end{eqnarray}

A portion of the spectrum centered 5373.71\,\AA\ is shown in Figure~\ref{fig:fig1}, where 
the true continuum is visible. The flux counts $y_{n}$ in this portion of spectrum are
shown as a histogram in Figure~\ref{fig:fig2}, where we also show the probability
distribution function for the foreground and background models given the true
continuum parameters ($c_0$, $c_1$, $c_2$), $A = 0$ indicating no absorption
(so $\mu_f = 5373.71$\,\AA\ and $\sigma_f = 0.1$ have no effect), $Q = 0.5$ and
$\sigma_g = 2$.

In Figure~\ref{fig:fig2} we show a histogram of the flux counts $y_{n}$ of this region.



\begin{figure*}
	\includegraphics[width=1.0\textwidth]{fig1.pdf}
    \caption{A portion of a `model' Solar spectrum (black) multiplied by a low-order 
    		 polynomial (blue) to represent the continuum.}
    \label{fig:fig1}
\end{figure*}

\textbf{Show the PDF of these two models for some values.}

\textbf{Model parameters and priors.}




\acknowledgments


\vspace{5mm}
%\facilities{LAMOST, Kepler, Gaia}

\software{
     \project{AstroPy}\ \citep{Astropy_2013,Astropy_2018},
     \project{numpy}\ \citep{Van-Der-Walt_2011},
     \project{scipy}\ \citep{Jones_2001},
     \project{matplotlib}\ \citep{Hunter_2007}
}    
     
\bibliographystyle{aasjournal}
\bibliography{bs}



%% Include this line if you are using the \added, \replaced, \deleted
%% commands to see a summary list of all changes at the end of the article.
%\listofchanges

\end{document}
